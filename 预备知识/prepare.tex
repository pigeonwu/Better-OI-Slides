\documentclass[10pt]{beamer}
\usetheme{Berlin}

\usepackage{ctex}
\usepackage{graphicx, graphics}
\usepackage{float, array, color, ctex}
\usepackage{amsmath, amssymb, amsfonts}
\usepackage{multicol, multirow, makecell, tabu, dcolumn}
\usepackage{fancyhdr, lastpage}
\usepackage{setspace}
\usepackage{geometry}

\usefonttheme[onlymath]{serif}

\geometry{papersize={144mm, 108mm}}

\setsansfont{苹方-简.ttf}
\setCJKsansfont{苹方-简.ttf}

\setstretch{1.25}

\AtBeginSection[]
{
	\begin{frame}
		\frametitle{当前进度}
		\tableofcontents[currentsection]
	\end{frame}
}

\AtBeginSubsection[]
{
	\begin{frame}
		\frametitle{当前进度}
		\tableofcontents[currentsubsection]
	\end{frame}
}

% 下面的内容会在标题页上展示
\title{算法竞赛预备知识}
\author{wwh}
\institute{合肥一六八中学}
\date{\today}

\begin{document}% 下面都是要显示的内容
	\frame{\titlepage}% 显示标题页
	
	\begin{frame}
		\frametitle{目录}
		\tableofcontents % 这个命令将会显示目录
	\end{frame}

	\section{C 与 C++ 简介}

	\section{编译工具、环境变量、命令行}	
	
	\subsection{编译工具}
	\begin{frame}
		\frametitle{程序是如何运行的}
		有一套精妙的步骤使得我们的程序跑起来。
		
		预处理$\to$编译$\to$汇编$\to$链接(这一套东西现阶段不需要理解,有兴趣的同学可以看看《深入理解计算机系统(CS:APP)》)
		
		\pause
		
		以上步骤都在 C++ 语言中,都通过 GCC 中的 gcc/g++ 完成。
	\end{frame}
	\begin{frame}
		\frametitle{拓展知识}
		\begin{block}{GNU, GCC, gcc 与 g++ 的简要区别与联系}
			\begin{itemize}
				\item<1-> GNU(GNU's Not Unix) 是 1984 年 Richard Stallman 发起的免税慈善项目,首次提出了自由软件(free software)的概念。
				\item<2-> GCC(GNU Compiler Collection) 是 GUN 的编译器集合,可以编译 C、C++、Pascal 等语言。
				\item<3-> gcc(GNU C Compiler) 是 GCC 中的 C 编译器。
				\item<3-> g++(GNU C++ Compiler) 是 GCC 中的 C++编译器。
				\item<4-> gcc 把 C 和 C++ 程序分开编译,且对于 C++ 文件不会自动链接 STL;g++ 对于 C 和 C++ 程序都按照 C++ 程序编译,且自动链接 STL(在之后的学习中会学到 STL)。
			\end{itemize}
		\end{block}
		
		\pause
		\pause
		\pause
		
		
		我们在 OI 学习中使用 g++ 编译程序!
	\end{frame}
	
	
	\subsection{环境变量}
	\begin{frame}
		\frametitle{什么是环境变量}
		系统环境中的变量。
	\end{frame}
	
	\subsection{命令行}
	
	\section{代码编辑器}
	
	\section{杂项}
	
\end{document}